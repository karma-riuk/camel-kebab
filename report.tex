\documentclass{article}
\usepackage{pgfplots}
\usepackage{multirow}
\usepackage{rotating}
\usepackage[]{float} 
\usepackage{hyperref} 
\usepackage[a4paper, total={6in, 8in}]{geometry}
\title{The Bubble Algorithm Library}
\author{Experiment 2, Experimentation \& Evaluation, 2022\\Arnaud Fauconnet,
	Francesco Costa}
\date{}
\pgfplotsset{compat=1.18}

\usepackage[most]{tcolorbox}


\newtcbtheorem{Hypotheses}{\bfseries Hypotheses}{enhanced,drop shadow={black!50!white},
	coltitle=white,
	top=0.3in,
	attach boxed title to top left=
		{xshift=1.5em,yshift=-\tcboxedtitleheight/2},
	boxed title style={size=small,colback=black}
}{summary}


\begin{document}
\maketitle
\tableofcontents

\section*{Abstract}


\section{Introduction}



\begin{Hypotheses*}{Best performing algorithm}{}
	Given that all the algorithms perform a subtle variation of the infamously
	known bubble sort algorithm, we do not expect to observe any significant
	difference in performance.
\end{Hypotheses*}

\section{Method}
\subsection{Variables}

\begin{description}
	\item[] 
\end{description}


\subsection{Design}


\subsection{Participants}

\subsection{Apparatus and Materials}
\label{apparatus}

\subsection{Procedure}

\section{Results}

\subsection{Visual Overview}

\subsection{Descriptive Statistics}
\label{stats}

\section{Discussion}

\subsection{Compare Hypothesis to Results}

\subsection{Limitations and Threats to Validity}

\subsection{Conclusions}

\end{document}
